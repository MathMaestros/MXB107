%!TEX TS-program = xelatex
%!TEX options = -aux-directory=Debug -shell-escape -file-line-error -interaction=nonstopmode -halt-on-error -synctex=1 "%DOC%"
\documentclass{article}
\input{LaTeX-Submodule/template.tex}

% Additional packages & macros

% Header and footer
\newcommand{\unitName}{Introduction to Statistical Modelling}
\newcommand{\unitTime}{Semester 2, 2022}
\newcommand{\unitCoordinator}{Dr Gentry White}
\newcommand{\documentAuthors}{Tarang Janawalkar}

\fancyhead[L]{\unitName}
\fancyhead[R]{\leftmark}
\fancyfoot[C]{\thepage}

% Copyright
\usepackage[
    type={CC},
    modifier={by-nc-sa},
    version={4.0},
    imagewidth={5em},
    hyphenation={raggedright}
]{doclicense}

\date{}

\begin{document}
%
\begin{titlepage}
    \vspace*{\fill}
    \begin{center}
        \LARGE{\textbf{\unitName}} \\[0.1in]
        \normalsize{\unitTime} \\[0.2in]
        \normalsize\textit{\unitCoordinator} \\[0.2in]
        \documentAuthors
    \end{center}
    \vspace*{\fill}
    \doclicenseThis
    \thispagestyle{empty}
\end{titlepage}
\newpage
%
\tableofcontents
\newpage
%
\section{Introduction}
Statistics is a field of mathematics that deals with data.
It includes the study of summarising data, constructing probabilistic models, estimating
parameters, and making statistical inferences.

Statistical modelling includes asking questions, obtaining data and determining a mathematical model.
\subsection{Elements of Statistical Modelling}
\subsubsection{Data}
Data is a collection of numbers that describes some characteristic that can be ranked, counted, or measured.
\subsubsection{Collecting information}
Statistical modelling relies upon reliably sourced data. When collecting data,
we must consider
\begin{itemize}
    \item what questions are we trying to answer,
    \item what information is needed to answer these questions,
    \item what is the best source for that information
\end{itemize}
\subsubsection{Randomness}
We must be aware that everything is different and that
randomness introduces uncertainty in data.
Random events are events whose exact outcome cannot be predicted.
We can assume that all variation in the world is observed due to randomness.
\subsubsection{Probability}
Probability is a mathematical construct for dealing with randomness and uncertainty.
\subsection{Experimental Units and Populations}
\begin{definition}[Experimental unit]
    An \textbf{experimental unit} is an individual that generates information for the data collection process.
    Careful consideration of what constitutes an \linebreak experimental unit must be made to ensure that it aligns with the questions of interest.
\end{definition}
\subsubsection{Sample vs. Population}
\begin{definition}[Population]
    We might have questions about a very large collection of things called a \textbf{population}.

    A dataset collected from a population is called a census.
\end{definition}
As it is not feasible to collect data from an entire population,
we must use a sample of the population.
\begin{definition}[Sample]
    A \textbf{sample} is a subset of a population that is representative of the population, in some cases a random sample is sufficient.
\end{definition}
\begin{definition}[Random sample]
    A \textbf{random sample} is one where the sample members are selected from the population by chance.
\end{definition}
\subsection{Types of Data}
\subsubsection{Univariate, Bivariate, and Multivariate}
Data can be described in terms of dimension, that is, how many measurements were collected from each experimental unit.
By collecting multiple measurements from each experimental unit, we can ask questions about the relationship between the measurements.
\begin{itemize}
    \item When a single measurement is collected, the resulting dataset is \textbf{univariate}.
    \item If two measurements are collected, the dataset is \textbf{bivariate}.
    \item If more than two measurements are collected, the dataset is \textbf{multivariate}.
\end{itemize}
\subsubsection{Experimental vs. Observational Data}
Data sets that have been collected without any specific analyses or modelling in mind are called \textbf{observational data}.
By contrast, when a collection procedure is specifically designed to obtain data with a specific intent,
i.e., a laboratory test, the data is called \textbf{experimental data}.

Observational data may contain biases that limit its usefulness and bias any modelling or analysis results.
\subsubsection{Quantitative Data}
Quantitative data is data that is expressed numerically.
This data can be classified as \textit{discrete}, \textit{continuous}, or \textit{ordinal}.
\begin{itemize}
    \item Count data is classified as discrete, i.e., integer values or finite sets
    of real values.
    \item Continuous data is a measurement on a continuum or a measure that can be subdivided infinitely,
    i.e., time and lengths.
    \item Ordinal data is data where the order or ranking of values (discrete or continuous) is important.
\end{itemize}
When data is not ordinal, it is called \textbf{nominal} data.
\subsubsection{Qualitative Data}
Qualitative (categorical) data is data where the variable of interest is
membership to a group or category.
\subsection{Summarising and Describing Data}
\subsubsection{Tables}
Tables are the most immediate way of summarising a data set.
We might organise data in a table with one row for each subject and a column for each measurement.
\subsection{Bar Charts}
Graphical depictions of the data can also be useful but are limited in the number of variables displayed in one picture.

Bar charts are most useful for categorical data where categories are listed on the \(x\)-axis of the plot, and bars for each category are drawn with their heights corresponding to the \textit{counts} for that category.

When the categories are \textbf{ordered} from left to right in descending order counts, the plot is called a \textbf{Pareto plot}.
\subsection{Line Charts}
Line charts illustrate a \textit{trend} of change based on \textbf{two} quantitative variables. Typically line charts display trends over time (or other ordinal variables).

Often trends over time need to be aggregated by plotting the average or median per year to avoid a ``busy'' plot which can sometimes be difficult to read.

While the resulting chart can explain overall trends, they can obscure how much variability or ``noise''
is in the data and may be misleading if the overall trend is obscured by variability.
\subsection{Histograms}
Histograms are a special kind of bar chart that give a visual description of data by ``binning'' or grouping data into data ranges,
then plotting bars with heights equal to the count of the bins' contents \textit{or} the relative proportion of the bins' contents.

Histograms give us a picture of the shape of the data and help identify patterns in the distribution of values.

The binning process is performed by the computer, however in most cases we override the automatic settings and select either the number of
bins, or the width of each bin.
\subsection{Plots, Graphs, and Charts}
\begin{itemize}
    \item A \textbf{chart} is a visual display of data, i.e., a table, a graph, or a diagram
    \item A \textbf{graph} is a diagram showing the relationship between variables, each measured along orthogonal axes.
    \item A \textbf{plot} is used as a synonym for graph but is less precise in its definition; it also sometimes refers specifically to a graph \textit{produced by a computer}.
\end{itemize}
\subsection{Interpreting Graphical Descriptions}
Graphical descriptions of data should ensure that all information about the data is expressed.
\begin{itemize}
    \item The \(x\) and \(y\) axes should be clear in what they are measuring, including any units.
    \item Consider how the graph or chart was made. What choices were made and how might different options change how the graph is perceived.
    \item Does the graph contain any outliers that merit investigation to determine if they are accurate measurements, or if they result from either measurement or recording error.
    \item For Pareto charts and histograms; the \(y\)-axis should measure proportion or density rather than frequency to make comparisons easier.
\end{itemize}
\subsubsection{Centrality}
Histograms are a graphical representation of the distribution or density of observations. Centrality is the degree to which an observation is central to the distribution. Additionally, the data can be multi-modal if there are multiple ``peaks'' or ``centres'' in the distribution.

Altering the number of bins or bin width may reveal the centrality of the observations.
\subsubsection{Skew}
Another characteristic of histograms is the degree to which the distribution is skewed. Skew is the deviation from symmetry about the centre of the data. Skew is either ``right'' skew where the tail of the density or histogram is heavier to the right, or ``left'' skew if otherwise.

This can be observed by looking at how much the left/right tails are stretched in comparison to one another, i.e., the tail to the right of a right skewed chart stretches further on the \(x\)-axis than on the left.
\subsubsection{Trends}
Trends refer to changes in a line chart and are often described as a
constant (first-derivative) pattern of increasing or decreasing values.
\end{document}
