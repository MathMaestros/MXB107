%!TEX TS-program = xelatex
%!TEX options = -aux-directory=Debug -shell-escape -file-line-error -interaction=nonstopmode -halt-on-error -synctex=1 "%DOC%"
\documentclass{article}
\input{LaTeX-Submodule/template.tex}

% Additional packages & macros
\usepackage{mathdots}
\setitemize{leftmargin=*,topsep=1ex,partopsep=0ex,itemsep=-1ex,partopsep=0ex,parsep=1ex}
\setlist{leftmargin=*,topsep=1ex,partopsep=0ex,itemsep=-1ex,partopsep=0ex,parsep=1ex}

\usepackage{changepage} % Modify page width
\usepackage{multicol} % Use multiple columns
\usepackage[explicit]{titlesec} % Modify section heading styles

\titleformat{\section}{\raggedright\normalfont\bfseries}{}{0em}{#1}
\titleformat{\subsection}{\raggedright\normalfont\small\bfseries}{}{0em}{#1}

%% A4 page
\geometry{
a4paper,
margin = 10mm
}

%% Hide horizontal rule
\renewcommand{\headrulewidth}{0pt}
\fancyhead{}

%% Hide page numbers
\pagenumbering{gobble}

%% Multi-columns setup
\setlength\columnsep{4pt}

%% Paragraph setup
\setlength\parindent{0pt}
\setlength\parskip{0pt}

%% Customise section heading styles
% \titleformat*\section{\raggedright\bfseries}

\begin{document}
\begin{multicols}{3}
    \section{Introduction}
    \subsection{Population}
    The entire group we are concerned with.
    \subsection{Sample}
    A representative subset of the population.
    \subsection{Quantitative Data}
    Numerical data.
    Could be nominal (discrete or continuous), or ordinal (ordered).
    \subsection{Qualitative Data}
    Categorical data, e.g.\ colour, model.

\section{Measures of Centrality}
\subsection{Mean}
Given a set of \(n\) observations \(x_1,\: x_2,\: \ldots,\: x_n\), the \textbf{arithmetic mean}
or \textbf{average} is defined as
\begin{equation*}
    \frac{1}{n} \sum_{i = 1}^n x_i
\end{equation*}
The sample mean is denoted \(\overline{x}\).
The population mean is denoted \(\mu\).
\subsection{Median}
A drawback to the mean is that it can be misleading when the data is skewed.
The \textbf{median} is the middle value of a set of \(n\) observations when arranged
from largest to smallest.

If \(n\) is odd:
\begin{equation*}
    \text{median} = x^{\left( \frac{n + 1}{2} \right)}
\end{equation*}
or the \(\left( n + 1 \right) / 2\)th value of the sorted list.
If \(n\) is even, the median is the :
\begin{equation*}
    \text{median} = \frac{x^{\left( \frac{n}{2} \right)} + x^{\left( \frac{n}{2} + 1 \right)}}{2}
\end{equation*}
\subsection{Mode}
Given discrete data, the mode is defined as the most common value in a set of observations.
\section{Measures of Dispersion}
Dispersion refers to how much variation there is in a set of observations.
\subsection{Range}
The range is the difference between the maximum and minimum observation.
\subsection{Variance}
The variance is the average of the squared deviations from the mean.
\begin{itemize}
    \item Given the observations \(x_1,\: x_2,\: \ldots,\: x_N\), from a population of size \(N\) with mean \(\mu\),
          the \textbf{population variance} is defined as
          \begin{equation*}
              \sigma^2 = \frac{1}{N} \sum_{i = 1}^N (x_i - \mu)^2.
          \end{equation*}
    \item Given the observations \(x_1,\: x_2,\: \ldots,\: x_n\), from a sample of size \(n\) with mean \(\overline{x}\),
          the \textbf{sample variance} is defined as
          \begin{equation*}
              s^2 = \frac{1}{n - 1} \sum_{i = 1}^n (x_i - \overline{x})^2.
          \end{equation*}
\end{itemize}
\subsection{Standard Deviation}
The standard deviation is the square root of the variance.
The \textbf{population standard deviation} is defined as
\(
              \sigma = \sqrt{\sigma^2}.
              \)
The \textbf{sample standard deviation} is defined as
\(
              s = \sqrt{s^2}.
              \)
\begin{theorem}[Chebyshev's Theorem]
    Given a set of \(n\) observations, at least
    \begin{equation*}
        1 - \frac{1}{k^2}
    \end{equation*}
    of them are within \(k\) standard deviations of the mean,
    where \(k \geq 1\).
\end{theorem}
\begin{theorem}[Empirical Rule]
    If a histogram of the data is approximately unimodal and symmetric, then,
    \begin{itemize}
        \item 68\% of the data falls within \textbf{one} standard deviation of the mean
        \item 95\% of the data falls within \textbf{two} standard deviations of the mean
        \item 99\% of the data falls within \textbf{three} standard deviations of the mean
    \end{itemize}
\end{theorem}
Often the standard deviation cannot be computed directly, but can be approximated
using the Empirical rule. Here we assume that
\begin{equation*}
    \text{range} \approx 4 s
\end{equation*}
so that
\begin{equation*}
    s = \frac{\text{range}}{4}.
\end{equation*}
\section{Skew}
The \textbf{skew} describes the asymmetry of the distribution.
For a finite population of size \(N\), the \textbf{population skew} is defined as
\begin{equation*}
    \frac{1}{N} \sum_{i = 1}^N \left( \frac{x_i - \mu}{\sigma} \right)^3
\end{equation*}
For a sample of size \(n\), the \textbf{sample skew} is defined as
\begin{equation*}
    \frac{\frac{1}{n} \sum_{i = 1}^n \left( x_i - \overline{x} \right)^3}{\left( \sqrt{\frac{1}{n - 1} \sum_{i = 1}^n \left( x_i - \overline{x} \right)^2} \right)^3}
\end{equation*}
\begin{itemize}
    \item When the skew is \textbf{positive}, the data is \textbf{right-skewed} and the ``tail'' of the distribution is \textbf{longer on the right}
    \item When the skew is \textbf{negative}, the data is \textbf{left-skewed} and the ``tail'' of the distribution is \textbf{longer on the left}
\end{itemize}
\section{Measures of Rank}
\subsection{Z-Score}
The Z-score is a unitless quantity and can be used to make comparisons of relative rank between members of a population.
\begin{equation*}
    Z = \frac{x  - \mu}{\sigma} \quad \text{or} \quad \frac{x - \overline{x}}{s}
\end{equation*}
\subsection{Quantiles}
For a set of \(n\) observations, \(x_q\) is the \(q\)-th quantile, if \(q\)\% of the observations are less than \(x_q\).
\subsection{Inter-Quartile Range}
The inter-quartile range (IQR) is the difference between the 75th and 25th quantiles,
or the range covered by the middle 50\% of data.

\subsection{Covariance and Correlation Coefficients}
Covariance is the measure of the linear correlation between variables.
\begin{equation*}
    s_{xy} = \frac{\sum_{i = 1}^n \left( x_i - \overline{x} \right) \left( y_i - \overline{y} \right)}{n - 1}.
\end{equation*}
Note that when \(x = y\), the formula simplifies to the sample variance of \(x\).
The covariance has the following characteristics:
\begin{itemize}
    \item \(s_{xy} > 0\): As \(x\) increases, \(y\) also increases.
    \item \(s_{xy} < 0\): As \(x\) increases, \(y\) decreases.
    \item \(s_{xy} \approx 0\): No relationship between \(x\) and \(y\).
\end{itemize}
\textbf{Correlation Coefficient}
\begin{equation*}
    -1 \le r_{xy} = \frac{s_{xy}}{s_x s_y} \le 1
\end{equation*}

Note that a correlation coefficient of \(0\)
indicates \textbf{no linear relationship} between the variables, and not necessarily indicative of \textbf{no relationship}.
\section{Regression and Least Squares}
A linear relationship between two variables \(x\) and \(y\) is defined as
\(
    y = a + b x.
    \)
The least squares best fit determines the coefficients \(a\) and \(b\) that minimise the sum of the squares of the residuals
\begin{align*}
    b & = r \frac{s_y}{s_x} = \frac{s_{xy}}{s_x^2} \\
    a & = \overline{y} - b \overline{x}.
\end{align*}


    \section{Events and Probability}
    % \subsection{Event}
    % Set of outcomes from an experiment.
    % \subsection{Sample Space}
    % Set of all possible outcomes \(\Omega\).
    % \subsection{Intersection}
    % Outcomes occur in both \(A\) and \(B\)
    % \begin{equation*}
    %     A \cap B \quad\quad \text{or} \quad\quad AB
    % \end{equation*}
    % \subsection{Disjoint}
    % No common outcomes, \(AB = \varnothing\)
    % \begin{equation*}
    %     \Pr{\left( AB \right)} = \Pr{\left( A \,\vert\, B \right)} = 0
    % \end{equation*}
    % \subsection{Union}
    % Set of outcomes in either \(A\) or \(B\)
    % \begin{equation*}
    %     A \cup B
    % \end{equation*}
    % \subsection{Complement}
    % Set of all outcomes not in \(A\), but in \(\Omega \)
    % \begin{align*}
    %     A\overline{A}       & = \varnothing \\
    %     A \cup \overline{A} & = \Omega
    % \end{align*}
    % \subsection{Subset}
    % \(A\) is a (non-strict) subset of \(B\) if all elements in \(A\) are also in \(B\) --- \(A \subset B\).
    % \begin{equation*}
    %     AB = A \quad\quad \text{and} \quad\quad A \cup B = B
    % \end{equation*}
    % \begin{equation*}
    %     \forall A:A\subset \Omega \land \varnothing \subset A
    % \end{equation*}
    % \begin{align*}
    %     \Pr{\left( A \right)}             & \leq \Pr{\left( B \right)}                            \\
    %     \Pr{\left( B \,\vert\, A \right)} & = 1                                                   \\
    %     \Pr{\left( A \,\vert\, B \right)} & = \frac{\Pr{\left( A \right)}}{\Pr{\left( B \right)}}
    % \end{align*}
    % \subsection{Identities}
    % \begin{align*}
    %     A \left( BC \right)            & = \left( AB \right) C                             \\
    %     A \cup \left( B \cup C \right) & = \left( A \cup B \right) \cup C                  \\
    %     A \left(B \cup C\right)        & = AB \cup AC                                      \\
    %     A \cup BC                      & = \left( A \cup B \right) \left( A \cup C \right)
    % \end{align*}
    % \subsection{Probability}
    % Measure of the likeliness of an event occurring
    % \begin{equation*}
    %     \Pr{\left( A \right)} \quad\quad \text{or} \quad\quad \mathrm{P}\left( A \right)
    % \end{equation*}
    % \begin{equation*}
    %     0 \leq \Pr{\left( A \right)} \leq 1
    % \end{equation*}
    % where a probability of 0 never happens, and 1 always happens.
    % \begin{align*}
    %     \Pr{\left( \Omega \right)}       & = 1                         \\
    %     \Pr{\left( \overline{A} \right)} & = 1 - \Pr{\left( A \right)}
    % \end{align*}
    \subsection{Multiplication Rule}
    For independent events \(A\) and \(B\)
    \begin{equation*}
        \Pr{\left( AB \right)} = \Pr{\left( A \right)} \Pr{\left( B \right)}.
    \end{equation*}
    For dependent events \(A\) and \(B\)
    \begin{equation*}
        \Pr{\left( AB \right)} = \Pr{\left( A \,\vert \, B \right)} \Pr{\left( B \right)}
    \end{equation*}
    \subsection{Addition Rule}
    For independent \(A\) and \(B\)
    \begin{equation*}
        \Pr{\left( A \cup B \right)} = \Pr{\left( A \right)} + \Pr{\left( B \right)} - \Pr{\left( AB \right)}.
    \end{equation*}
    If \(AB = \varnothing \), then \(\Pr{\left( AB \right)} = 0\), so that \(\Pr{\left( A \cup B \right)} = \Pr{\left( A \right)} + \Pr{\left( B \right)}\).
    \subsection{De Morgan's Laws}
    \begin{align*}
        \overline{A \cup B} & = \overline{A} \ \overline{B}     \\
        \overline{AB}       & = \overline{A} \cup \overline{B}.
    \end{align*}
    \begin{align*}
        \Pr{\left( A \cup B \right)} & = 1 - \Pr{\left( \overline{A} \ \overline{B} \right)}    \\
        \Pr{\left( AB \right)}       & = 1 - \Pr{\left( \overline{A} \cup \overline{B} \right)}
    \end{align*} 
    \subsection{Bayes' Theorem}
    \begin{equation*}
        \Pr{\left( A \,\vert\, B \right)} = \frac{\Pr{\left( B \,\vert\, A \right)}\Pr{\left( A \right)}}{\Pr{\left( B \right)}}
    \end{equation*}
    \section{Random Variables}
        Measurable variable whose value holds some uncertainty.
        An event is when a random variable assumes a certain value or range of values.
        \subsection{Probability Distribution}
        The probability distribution of a random variable \(X\) is a function that links all outcomes \(x \in \Omega\)
        to the probability that they will occur \(\Pr{\left( X = x \right)}\).
        \subsection{Probability Mass Function}
        \begin{equation*}
            \Pr{\left( X = x \right)} = p_x
        \end{equation*}
        \subsection{Probability Density Function}
        \begin{equation*}
            \Pr{\left( x_1 \leq X \leq x_2 \right)} = \int_{x_1}^{x_2} f\left( x \right) \odif{x}
        \end{equation*}
        \subsection{Cumulative Distribution Function}
        Probability that a random variable is
        less than or equal to a particular realisation \(x\).

        \(F\left( x \right)\) is a valid CDF if:
        \begin{enumerate}
            \item \(F\) is monotonically increasing and continuous
            \item \(\lim_{x \to -\infty} F\left( x \right) = 0\)
            \item \(\lim_{x \to \infty} F\left( x \right) = 1\)
        \end{enumerate}
        \begin{equation*}
            \odv{F\left( x \right)}{x} = \odv{}{x} \int_{-\infty}^x f\left( u \right) \odif{u} = f\left( x \right)
        \end{equation*}
        \subsection{Complementary CDF (Survival Function)}
        \begin{equation*}
            \Pr{\left( X > x \right)} = 1 - \Pr{\left( X \leq x \right)} = 1 - F\left( x \right)
        \end{equation*}
        \subsection{\texorpdfstring{\(p\)}{p}-Quantiles}
        \begin{equation*}
            F\left( x \right) = \int_{-\infty}^x f\left( u \right) \odif{u} = p
        \end{equation*}
        \subsection{Special Quantiles}
        \begin{align*}
            \text{Lower quartile \(q_1\):}  &  &  & p = \frac{1}{4} \\
            \text{Median \(m\):}            &  &  & p = \frac{1}{2} \\
            \text{Upper quartile \(q_2\):}  &  &  & p = \frac{3}{4} \\
            \text{Interquartile range IQR:} &  &  & q_2 - q_1
        \end{align*}
        \subsection{Quantile Function}
        \begin{equation*}
            x = F^{-1}\left( p \right) = Q\left( p \right)
        \end{equation*}
        \subsection{Expectation (Mean)}
        Expected value given an infinite number of observations. For \(a < c < b\):
        \begin{equation*}
            \E{\left(X\right)} = \begin{aligned}[t]
                 & -\int_{a}^c F\left( x \right) \odif{x}                     \\
                 & + \int_c^b \left(1 - F\left( x \right)\right) \odif{x} + c
            \end{aligned}
        \end{equation*}
        \subsection{Variance}
        Measure of spread of the distribution (average squared distance of each value from the mean).
        \begin{equation*}
            \Var{\left( X \right)} = \sigma^2 = \E{\left( X^2 \right)} - \E{\left( X \right)}^2
        \end{equation*}
        \subsection{Standard Deviation}
        \begin{equation*}
            \sigma = \sqrt{\Var{\left( X \right)}}
        \end{equation*}
        

\subsection{Central Limit Theorem}
For a sample of size \(n\) from any random probability distribution with expected value \(\mu\)
and variance \(\sigma^2\),
\begin{equation*}
    \frac{\sqrt{n}\left( \overline{x} - \mu \right)}{\sigma} \overset{p}{\rightarrow} \operatorname{N}{\left( 0,\: 1 \right)}
\end{equation*}
meaning that increasing the sample size will lead to a more normal distribution.
In this case, a sample size of \(n = 30\) is sufficient to approximate a normal distribution.
\subsection{Standard Error}
\begin{equation*}
    \operatorname{SE}{\left( \overline{x} \right)} = \frac{\sigma^2}{n}
\end{equation*}
\subsection{Sample Proportion}
For a sample of size \(n\) let \(x\) be the number of members with a particular characteristic.
The sample estimate of the population proportion \(p\) is
\begin{equation*}
    \hat{p} = \frac{x}{n}.
\end{equation*}
By assuming that the samples statistic \(x\) follows a binomial distribution with probability \(p\) and
size \(n\), then \(\E{\left( x \right)} = np\) and \(\Var{\left( x \right)} = np\left( 1 - p \right)\).
Therefore the expectation is
\begin{equation*}
    \E{\left( \hat{p} \right)} = p
\end{equation*}
and the standard error is
\begin{equation*}
    \operatorname{SE}{\left( \hat{p} \right)} = \sqrt{\frac{p\left( 1 - p \right)}{n}}.
\end{equation*}
For the above to apply, we must assume that the sample proportion and size are sufficiently large.
In general, if \(np > 5\) and \(n\left( 1 - p \right) > 5\), then we can assume that the sampling distribution of \(\hat{p}\)
is approximately normal.
\subsection{Assessing Normality}
\begin{itemize}
    \item Histograms: if the data is approximately normal, then the histogram will be approximately symmetric and unimodal.
    \item Boxplots: boxplots can be useful for showing outliers and skewness. Extreme clusters of an excessive number of
          outliers can be evidence of non-normality.
    \item Normal probability plots (\(q\)-\(q\) plots): these plots are constructed by plotting the sorted data values against
          their \(Z\)-scores. If the data is approximately normal, then the points will lie approximately on a straight line.
\end{itemize}
\section{Large Sample Estimation}
\subsection{Point Estimation}
\subsection{Method of Moments}
The moments of a probability distribution are defined
\begin{align*}
    \mu_n & = \E{\left( X^n \right)} = \int_{-\infty}^\infty x^n f\left( x \right) \odif{x}
\end{align*}
where \(f\left( x \right)\) is the probability density function of the distribution.
Here \(\mu_1 = \E{\left( X \right)}\) and \(\Var{\left( X \right)} = \mu_2 - \mu_1^2\).
Sample moments are defined similarly
\begin{align*}
    m_n & = \frac{1}{n} \sum_{i = 1}^n x_i^n
\end{align*}
where \(\overline{x} = m_1\).
\subsection{Method of Maximum Likelihood Estimation}
\begin{definition}[Likelihood function]
    \begin{equation*}
        \mathcal{L}\left( \theta \,\vert\, \symbfit{x} \right) = \prod_{i = 1}^n f\left( x_i \right) \\
    \end{equation*}
\end{definition}
\begin{definition}[Maximum likelihood estimator]
    \begin{equation*}
        \hat{\theta} = \arg\max_\theta{\mathcal{L}\left( \theta \,\vert\, \symbfit{x} \right)}.
    \end{equation*}
\end{definition}
\begin{definition}[Log-likelihood function]
    The \textbf{log-likelihood function} is defined as
    \begin{align*}
        \ell\left( \theta \,\vert\, \symbfit{x} \right)& = \sum_{i = 1}^n \log{\left( f\left( x_i \right) \right)}
    \end{align*}
\end{definition}
Due to the monotonicity of the log function, the maximum likelihood estimator is the same as the maximum log-likelihood estimator.
\begin{definition}[Maximum log-likelihood estimator]
    The \textbf{maximum log-likelihood estimator} is defined as
    \begin{equation*}
        \hat{\theta} = \arg\max_\theta{\ell\left( \theta \,\vert\, \symbfit{x} \right)}.
    \end{equation*}
\end{definition}
\subsection{Properties of Estimators}
\begin{definition}[Bias]
    The \textbf{bias} of an estimator is defined as
    the difference between the expected value of the estimator \(\E{\left( \hat{\theta} \right)}\) and the true value of the parameter \(\theta_0\).
    \begin{equation*}
        \operatorname{Bias}\left( \hat{\theta} \right) = \E{\left( \hat{\theta} \right)} - \theta
    \end{equation*}
    An estimator \(\hat{\theta}\) is \textbf{unbiased} if
    \begin{equation*}
        \E{\left( \hat{\theta} \right)} = \theta
    \end{equation*}
    so that the bias is zero.
\end{definition}
We can also compare the variance of two estimators, to assess which one is more preferable.
If the variance of the estimator is small, then the estimator is more precise.
Given two estimators \(\hat{\theta}_1\) and \(\hat{\theta}_2\), we would choose \(\hat{\theta}_1\) over \(\hat{\theta}_2\) if
\begin{equation*}
    \Var{\left( \hat{\theta}_1 \right)} < \Var{\left( \hat{\theta}_2 \right)}
\end{equation*}
\begin{definition}[Mean square error]
    Given data \(x_i\) with variance \(\sigma^2\), the estimators of \(\theta = \E\left\{ X \right\}\)
    are selected such that they minimise the \textbf{mean square error}:
    \begin{align*}
        \operatorname{MSE}\left( \hat{\theta} \right) & = \E{\left( \left( \hat{\theta} - \theta \right)^2 \right)}                             \\
                                                      & = \operatorname{Bias}\left( \hat{\theta} \right)^2 + \Var{\left( \hat{\theta} \right)}.
    \end{align*}
    This quantity is used to determine the \textbf{bias-variance trade-off} of an estimator.
    The \textbf{root mean square error} is defined as
    \begin{equation*}
        \operatorname{RMSE}\left( \hat{\theta} \right) = \sqrt{\operatorname{MSE}\left( \hat{\theta} \right)}.
    \end{equation*}
\end{definition}
\subsection{Confidence Intervals}
This interval ranges from the lower confidence limit (UCL) to the upper confidence limit (LCL)
\begin{equation*}
    L < \theta < U.
\end{equation*}
This interval has a \textbf{confidence coefficient} of \(1 - \alpha\), or a \textbf{confidence level} of \(\left( 1 - \alpha \right)\%\).
The confidence interval is defined as
\begin{align*}
    {CI}_{1 - \alpha} &= \hat{\theta} \pm Z_{\alpha / 2} \operatorname{SE}\left( \hat{\theta} \right)
\end{align*}
\subsection{Confidence Interval for the Mean}
\begin{equation*}
    {CI}_{1-\alpha} = \overline{x} \pm Z_{\alpha/2} \frac{\sigma}{\sqrt{n}}
\end{equation*}
where
\(
    \operatorname{SE}\left( \overline{x} \right) = \frac{\sigma}{\sqrt{n}}.
    \)
\subsection{Confidence Interval for the Proportion}
Given the sample size \(n\) and sample proportion \(\hat{p}\),
\begin{equation*}
    \hat{p} \sim \operatorname{N}\left( p,\: \frac{p\left( 1 - p \right)}{n} \right).
\end{equation*}
The confidence interval for the population proportion is
\begin{equation*}
    {CI}_{1-\alpha} = \hat{p} \pm Z_{\alpha/2} \sqrt{\frac{\hat{p}\left( 1 - \hat{p} \right)}{n}}
\end{equation*}
where the standard error is given by
\begin{equation*}
    \operatorname{SE}\left( \hat{p} \right) = \sqrt{\frac{p\left( 1 - p \right)}{n}}.
\end{equation*}
with the approximation \(p=\hat{p}\).
Note that \(n \hat{p} > 5\) and \(n \left( 1 - \hat{p} \right) > 5\) are required for the approximation to be valid.
\subsection{Confidence Interval for the Difference of Two Means}
% Given two population means \(\mu_1\) and \(\mu_2\), we expect the difference of the population means
% and the sample means to be equal. Consider the expectation of the difference of the sample means:
% \begin{equation*}
%     \E{\left( \overline{x}_1 - \overline{x}_2 \right)} = \mu_1 - \mu_2
% \end{equation*}
% with standard error
% \begin{equation*}
%     \operatorname{SE}\left( \overline{x}_1 - \overline{x}_2 \right) = \sqrt{\frac{\sigma_1^2}{n_1} + \frac{\sigma_2^2}{n_2}}
% \end{equation*}
% with is estimated by \(\sigma=s\).
% The confidence interval for the difference of the two means is
\begin{equation*}
    {CI}_{1-\alpha} = \overline{x}_1 - \overline{x}_2 \pm Z_{\alpha/2} \sqrt{\frac{s_1^2}{n_1} + \frac{s_2^2}{n_2}}.
\end{equation*}
If the two populations follow a normal distribution, then the sampling distribution is exactly normal.
If the two populations are not normal, then the sampling distribution is approximately normal, for \(n_1 > 30\) and \(n_2 > 30\).
\subsection{Confidence Interval for the Difference of Two Proportions}
\begin{equation*}
    {CI}_{1-\alpha} = \hat{p}_1 - \hat{p}_2 \pm Z_{\alpha/2} \sqrt{\frac{\hat{p}_1\left( 1 - \hat{p}_1 \right)}{n_1} + \frac{\hat{p}_2\left( 1 - \hat{p}_2 \right)}{n_2}}.
\end{equation*}
Note that the following constraints must be satisfied:
\begin{itemize}
    \item \(n_1 \hat{p}_1 > 5\)
    \item \(n_1 \left( 1 - \hat{p}_1 \right) > 5\)
    \item \(n_2 \hat{p}_2 > 5\)
    \item \(n_2 \left( 1 - \hat{p}_2 \right) > 5\)
\end{itemize}
\section{Hypothesis Testing}
\subsection{Hypothesis Testing for the Population Mean}
Given the sample statistic \(\overline{x}\),
\begin{equation*}
    \overline{x} \sim \operatorname{N}\left( \mu,\: \frac{\sigma^2}{n} \right)
\end{equation*}
the test statistic is defined
\begin{equation*}
    T\left( \symbfit{x} \right) = \frac{\overline{x} - \mu_0}{\sigma/\sqrt{n}} \sim \operatorname{N}\left( 0,\: 1 \right).
\end{equation*}
\subsection{Hypothesis Testing for the Population Proportion}
Given the sample statistic \(\hat{p}\),
\begin{equation*}
    \hat{p} \sim \operatorname{N}\left( p,\: \frac{p\left( 1 - p \right)}{n} \right)
\end{equation*}
for \(n \hat{p} > 5\) and \(n \left( 1 - \hat{p} \right) > 5\),
the test statistic is defined
\begin{equation*}
    T\left( \symbfit{x} \right) = \frac{\sqrt{n} \left( \hat{p} - p_0 \right)}{\sqrt{p_0 \left( 1 - p_0 \right)}}.
\end{equation*}
\subsection{Hypothesis Testing with Differences}
The rejection regions for the difference between two parameters is defined:
\begin{center}
    \begin{tabular}{cc}
        \toprule
        \textbf{Null Hypothesis} \(H_0\) & \textbf{Rejection Region} \(R\)                       \\
        \midrule
        \(\theta_1 - \theta_0 = 0\)      & \(\abs*{T\left( \symbfit{x} \right)} > Z_{\alpha/2}\) \\
        \(\theta_1 - \theta_2 \leq 0\)   & \(T\left( \symbfit{x} \right) > Z_{\alpha}\)          \\
        \(\theta_1 - \theta_2 \geq 0\)   & \(T\left( \symbfit{x} \right) < -Z_{\alpha}\)         \\
        \bottomrule
    \end{tabular}
\end{center}
\subsection{Hypothesis Testing for the Difference in Population Means}
The point estimator of \(\mu_1 - \mu_2\) is given by
\begin{equation*}
    \overline{x}_1 - \overline{x}_2
\end{equation*}
and the standard error is given by
\begin{equation*}
    \operatorname{SE}_{\overline{x}_1 - \overline{x}_2} = \sqrt{\frac{s_1^2}{n_1} + \frac{s_2^2}{n_2}}.
\end{equation*}
The test statistic is defined
\begin{equation*}
    T\left( \symbfit{x} \right) = \frac{\left( \overline{x}_1 - \overline{x}_2 \right) - \Delta_0}{\operatorname{SE}_{\overline{x}_1 - \overline{x}_2}}.
\end{equation*}
where \(\Delta_0 = \mu_1 - \mu_2\) is the hypothesized difference between the two population means.
\subsection{Hypothesis Testing for the Difference in Population Proportions}
The point estimator of the difference in proportions where \(p_1 = p_2\) is given by
\begin{equation*}
    \hat{p}_1 - \hat{p}_2
\end{equation*}
and the standard error is defined
\begin{equation*}
    \operatorname{SE}_{\hat{p}_1 - \hat{p}_2} = \sqrt{p_0 \left( 1 - p_0 \right) \left( \frac{1}{n_1} + \frac{1}{n_2} \right)}
\end{equation*}
where
\begin{align*}
    p_0 & = \frac{x_1 + x_2}{n_1 + n_2}                      \\
    p_0 & = \frac{\hat{p}_1 n_1 + \hat{p}_2 n_2}{n_1 + n_2}.
\end{align*}
so that \(p_0 = p_1 = p_2\).
The resulting test statistic is defined:
\begin{equation*}
    T\left( \symbfit{x} \right) = \frac{\left( \hat{p}_1 - \hat{p}_2 \right)}{\operatorname{SE}_{\hat{p}_1 - \hat{p}_2}}.
\end{equation*}
When the hypothesised difference is not 0, i.e., \(p_1 - p_2 = \Delta_0\), the test statistic is defined:
\begin{equation*}
    T\left( \symbfit{x} \right) = \frac{\left( \hat{p}_1 - \hat{p}_2 \right) - \Delta_0}{\sqrt{\frac{\hat{p}_1 \left( 1 - \hat{p}_1 \right)}{n_1} + \frac{\hat{p}_2 \left( 1 - \hat{p}_2 \right)}{n_2}}}.
\end{equation*}
\subsection{Significance of Results}
When interpreting the results from a test statistic, the test can only be used to reject the null hypothesis.
\section{Small Sample Inference}
\textbf{Student's t-distribution}:
\begin{equation*}
    T\left( \symbfit{x} \right) \sim t_{\nu}
\end{equation*}
where the degrees of freedom \(\nu\) is equal to \(n - 1\).
\begin{align*}
    \E{\left( X \right)}   & = 0                   \\
    \Var{\left( X \right)} & = \frac{\nu}{\nu - 2} \\
\end{align*}
\subsection{Inferencing}
\begin{equation*}
    T\left( \symbfit{x} \right) = \frac{\overline{x} - \mu_0}{s / \sqrt{n}} \sim t_{\nu, \alpha/2}.
\end{equation*}
\subsection{Hypothesis Testing for the Population Mean}
\begin{equation*}
    T\left( \symbfit{x} \right) = \frac{\overline{x} - \mu_0}{s/\sqrt{n}} \sim t_{\nu, \alpha/2}.
\end{equation*}
\subsection{Hypothesis Testing for the Difference in Population Means}
\begin{equation*}
    T\left( \symbfit{x} \right) = \frac{\overline{x}_1 - \overline{x}_2}{\sqrt{s^2\left( \frac{1}{n_1} + \frac{1}{n_2} \right)}} \sim t_{\nu, \alpha/2}.
\end{equation*}
If the sample variances \(s_1^2\) and \(s_2^2\) are not equal, then we
need to determine the common or \textit{pooled} variance \(s_p^2\).
\begin{equation*}
    s_p^2 = \frac{\left( n_1 - 1 \right)s_1^2 + \left( n_2 - 1 \right)s_2^2}{\nu}.
\end{equation*}
where \(\nu = n_1 + n_2 - 2\) for the two-sample \(t\)-test.
This results in the following test statistic:
\begin{equation*}
    T\left( \symbfit{x} \right) = \frac{\overline{x}_1 + \overline{x}_2}{\sqrt{s_p^2 \left( \frac{1}{n_1} + \frac{1}{n_2} \right)}}.
\end{equation*}
The population variances between two samples vary \textit{greatly}, if
they satisfy the following:
\begin{equation*}
    \frac{\max{\left( s_1^2,\: s_2^2 \right)}}{\min{\left( s_1^2,\: s_2^2 \right)}} > 3.
\end{equation*}
when this is the case, we must modify the test statistic to account for the
different variances:
\begin{equation*}
    T\left( \symbfit{x} \right) = \frac{\left( \overline{x}_1 - \overline{x}_2 \right) - \Delta_0}{\sqrt{\frac{s_1^2}{n_1} + \frac{s_2^2}{n_2}}}
\end{equation*}
noting that \(\Delta_0\) is typically zero. The degrees of freedom are given by
\begin{equation*}
    \nu = \floor*{\frac{\left( \frac{s_1^2}{n_1} + \frac{s_2^2}{n_2} \right)^2}{\frac{\left( s_1^2 / n_1 \right)^2}{n_1 - 1} + \frac{\left( s_2^2 / n_2 \right)^2}{n_2 - 1}}}
\end{equation*}
where the value is truncated (towards zero).

Note that the above only applies for independent samples.

        
\begin{minipage}{80mm}
\begin{table}[H]
    \centering
    \begin{tabular}{c|cc}
        \toprule
        \textbf{Decision}                  & \textbf{\(H_0\)}       & \textbf{\(\lnot H_0\)}      \\
        \midrule
        \textbf{Reject \(H_0\)}            & \(\alpha\)  & \(1 - \beta\)           \\
        \textbf{Fail to reject} & \(1 - \alpha\)                 & \(\beta\)  \\
        \bottomrule
    \end{tabular}
    % \caption{} % \label{}
\end{table}
\end{minipage}

\begin{minipage}{80mm}
\begin{table}[H]
    \centering
    \begin{tabular}{cc}
        \toprule
        \(H_0\) & \textbf{Rejection Region} \(R\)                       \\
        \midrule
        \(\theta = \theta_0\)            & \(\abs*{T\left( \symbfit{x} \right)} > Z_{\alpha/2}\) \\
        \(\theta \leq \theta_0\)         & \(T\left( \symbfit{x} \right) > Z_{\alpha}\)          \\
        \(\theta \geq \theta_0\)         & \(T\left( \symbfit{x} \right) < -Z_{\alpha}\)         \\
        \bottomrule
    \end{tabular}
    % \caption{} % \label{}
\end{table}
\end{minipage}

% Should be provided according to Slack
% \begin{minipage}{126.1962963mm}
% \begin{table}[H]
%     \centering
%     \begin{tabular}{ccc}
%         \toprule
%         \textbf{Type I Error Rate} \(\alpha\) & \textbf{One Tail} \(Z_{\alpha}\) & \textbf{Two-Tail} \(Z_{\alpha/2}\) \\
%         \midrule
%         \(0.10\)                              & \(1.28\)                         & \(1.64\)                           \\
%         \(0.05\)                              & \(1.65\)                         & \(1.96\)                           \\
%         \(0.02\)                              & \(2.05\)                         & \(2.33\)                           \\
%         \(0.01\)                              & \(2.33\)                         & \(2.58\)                           \\
%         \bottomrule
%     \end{tabular}
%     \caption{} % \label{}
% \end{table}
% \end{minipage}

\begin{minipage}{70mm}
    \begin{table}[H]
        \centering
        \scriptsize
        \begin{tabular}{c c c }
            \toprule
                                                     & \textbf{Discrete}                              & \textbf{Continuous}                                    \\
            \midrule
            \(\E{\left( X \right)}\)                 & \(\sum_{\Omega} xp_x\)                         & \(\int_{\Omega} xf(x)\odif{x}\)                        \\
            \(\E{\left( g\left( X \right) \right)}\) & \(\sum_{\Omega} g\left( x \right)p_x\)         & \(\int_{\Omega} g\left( x \right)f(x)\odif{x}\)        \\
            \(\Var{\left( X \right)}\)               & \(\sum_{\Omega} \left( x - \mu \right)^2 p_x\) & \(\int_{\Omega} \left( x - \mu \right)^2f(x)\odif{x}\) \\
            \bottomrule
        \end{tabular}
        % \caption{Probability rules for univariate \(X\).} % \label{}
    \end{table}
\end{minipage}
\end{multicols}

\begin{table}[H]
    \centering
    \begin{tabular}{c c c c c c}
        \toprule
        \textbf{Distribution}                                      & \textbf{Restrictions}                                  & \textbf{PMF}                                              & \textbf{CDF}                                                             & \(\E{\left( X \right)}\)            & \(\Var{\left( X \right)}\)                    \\
        \midrule
        \(X \sim \operatorname{Uniform}{\left( a,\: b \right)}\)   & \(x \in \left\{ a, \dots, b \right\}\)                 & \(\frac{1}{b - a + 1}\)                                   & \(\frac{x - a + 1}{b - a + 1}\)                                          & \(\frac{a + b}{2}\)                 & \(\frac{\left( b - a + 1 \right)^2 - 1}{12}\) \\
        \(X \sim \operatorname{Bernoulli}{\left( p \right)}\)      & \(p \in \interval{0}{1}, x \in \left\{ 0, 1 \right\}\) & \(p^x \left( 1 - p \right)^{1 - x}\)                      & \(1 - p\)                                                                & \(p\)                               & \(p \left( 1 - p \right)\)                    \\
        \(X \sim \operatorname{Binomial}{\left( n,\: p \right)}\)  & \(x \in \left\{ 0, \dots, n \right\}\)                 & \(\binom{n}{x} p^x \left( 1 - p \right)^{n - x}\)         & \(\sum_{u = 0}^x \binom{n}{u} p^u \left( 1 - p \right)^{n - u}\)         & \(np\)                              & \(np\left( 1 - p \right)\)                    \\
        \( N \sim \operatorname{Poisson}{\left( \lambda \right)}\) & \(n \geq 0\)                                           & \(\frac{\lambda^n e^{-\lambda}}{n!}\)                     & \(e^{-\lambda} \sum_{u = 0}^n \frac{\lambda^u}{u!}\)                     & \(\lambda\)                         & \(\lambda\)                                   \\
        \bottomrule
    \end{tabular}
    % \caption{Discrete probability distributions.} % \label{}
\end{table}
\begin{table}[H]
    \centering
    \begin{tabular}{c c c c c c}
        \toprule
        \textbf{Distribution}                                       & \textbf{Restrictions}                  & \textbf{PDF}                                                                         & \textbf{CDF}                                                                            & \(\E{\left( X \right)}\) & \(\Var{\left( X \right)}\)            \\
        \midrule
        \(X \sim \operatorname{Uniform}{\left( a,\: b \right)}\)    & \(a < x < b\)                          & \(\frac{1}{b - a}\)                                                                  & \(\frac{x - a}{b - a}\)                                                                 & \(\frac{a + b}{2}\)      & \(\frac{\left( b - a \right)^2}{12}\) \\
        \(T \sim \operatorname{Exp}{\left( \eta \right)}\)          & \(t > 0\)                              & \(\eta e^{-\eta t}\)                                                                 & \(1 - e^{-\eta t}\)                                                                     & \(1/\eta\)               & \(1/\eta\)                            \\
        \(X \sim \operatorname{N}{\left( \mu,\: \sigma^2 \right)}\) & \(x \in \left\{ 0, \dots, n \right\}\) & \(\frac{1}{\sqrt{2 \pi \sigma^2}} e^{-\frac{\left( x - \mu \right)^2}{2 \sigma^2}}\) & \(\frac{1}{2} \left( 1 + \erf{\left( \frac{x - \mu}{\sigma \sqrt{2}} \right)} \right)\) & \(\mu\)                  & \(\sigma^2\)                          \\
        \bottomrule
    \end{tabular}
    % \caption{Continuous probability distributions.} % \label{}
\end{table}
\end{document}
